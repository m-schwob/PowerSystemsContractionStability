%% LyX 2.3.6.1 created this file.  For more info, see http://www.lyx.org/.
%% Do not edit unless you really know what you are doing.
\documentclass[12pt,english,hebrew]{article}
\usepackage{amsmath}
\usepackage{amssymb}
\usepackage{fontspec}
\setmainfont[Mapping=tex-text]{Times New Roman}
\setsansfont[Mapping=tex-text]{Arial}
\setmonofont{Miriam}
\usepackage[a4paper]{geometry}
\geometry{verbose,tmargin=2cm,bmargin=2cm,lmargin=2cm,rmargin=2cm}
\setlength{\parindent}{0bp}
\usepackage{color}
\usepackage{mathtools}
\usepackage{url}

\makeatletter

%%%%%%%%%%%%%%%%%%%%%%%%%%%%%% LyX specific LaTeX commands.
\providecommand\textquotedblplain{%
  \bgroup\addfontfeatures{Mapping=}\char34\egroup}

%%%%%%%%%%%%%%%%%%%%%%%%%%%%%% Textclass specific LaTeX commands.
\usepackage{theorem}
\theorembodyfont{\upshape}
% LuaTeX/luabidi does not know \beginR
% FIXME LuaTeX/luabidi does not get the order right
\AtBeginDocument{
\@ifundefined{setRTL}{}{\providecommand\beginR{\setRTL}}
}
\newtheorem{theorem}{{\beginR משפט}}[section]
% Only needed by babel, not polyglossia (which does
% not have \make@lr defined)
\@ifundefined{make@lr}{}{%
   \AtBeginDocument{\make@lr\thetheorem}
}
\newtheorem{definition}[theorem]{{\beginR הגדרה}}

%%%%%%%%%%%%%%%%%%%%%%%%%%%%%% User specified LaTeX commands.
%%%% GENERAL %%%%

%\date{} %remove comment will disable the auto date on the document

%fix furmula numbering brackets.
\AtBeginDocument{
\makeatletter
\def\tagform@#1{\maketag@@@{(\ignorespaces#1\unskip)}}
\makeatother 
}


%%%% TIKZ %%%%

%general tikz drawing
\usepackage{tikz}
\usepackage{standalone}	%use to add standalone tikz picture tex file
%for easy coordinates calculation
\usepackage{calc}
\usetikzlibrary{calc}
%for drawing angles easly
%\usepackage{tkz-euclide} commpile collision with 'tikz-network'
%\usetkzobj{all} compile issue

%\usepackage{tikz-network} %for graph drawing
%for electronic circuits via tikz. siunitx is package for units.
%\usepackage[american,siunitx]{circuitikz}


%%%% STYLE %%%%

% manual colors setup
\definecolor{red}{rgb}{0.9,0,0}
\definecolor{orange}{rgb}{1,0.6,0}
\definecolor{yellow}{rgb}{1,0.8,0}
\definecolor{green}{rgb}{0,0.5,0.3}
\definecolor{blue}{rgb}{0,0.25,0.65}
\definecolor{purple}{rgb}{0.6,0,0.6}

\makeatother

\usepackage{listings}
\lstset{mathescape=true,
language=Python,
keywordstyle={\color{purple}\bfseries},
commentstyle={\color{magenta}\itshape},
emphstyle={\color{red}},
breaklines=true,
stringstyle={\color{green}},
identifierstyle={\color{blue}\ttfamily},
tabsize=4}
\usepackage{polyglossia}
\setdefaultlanguage{hebrew}
\setotherlanguage{english}
\begin{document}

\section{מרחבים מטרים (מטריקות)}

\subsection{מטריקות}

\paragraph{אינטואיציה מקדימה}

מטריקות הן הרחבה של צורת המדידה. למשל אפשר למדוד יחס בין שני נקודות
במרחב \LRE{$R^{3}$} ע\textquotedblplain י מתיחת קו ישר בניהם (המסלול
הקצר ביותר). זוהי צורת המדידה שאנחנו רגילים אליה. מטריקות באות להרחיב
את שיטות המידה לצורות נוספות של מדידה, למשל למדוד מרחק בין שני נקודות
ע\textquotedblplain י אורך קשת ברדיוס מסויים. בנוסף, מטריקות מרחיבות
את האובייקטים עליהם אחנו מבצעים מדידה, למשל מדידה של מרחק או יחס מסויים
בין פונקציות.

לסיכום, מטריקות מגדירות קשר/יחס כלשהו בין שני אובייקטים במרחב כלשהו.
נשים לב שמטריקות תמיד מחזירות סקלר.
\begin{definition}
\textit{מרחב מטרי} הוא זוג \LRE{$\left(X,\rho\right)$} של קבוצה לא
ריקה \LRE{$X$} ופונקציה (מטריקה) \LRE{$\rho:X\times X\rightarrow\mathbb{R}$}
כך ש-\LRE{$\rho$} מקיימת:
\begin{enumerate}
\item סימטריות: לכל \LRE{$x,y\in X$}, \LRE{$\rho\left(x,y\right)=\rho\left(y,x\right)$}.
\item חיוביות: לכל \LRE{$x,y\in X$}, \LRE{$\rho\left(x,y\right)\geq0$}
וגם \LRE{$\rho\left(x,y\right)=0$} אמ\textquotedblplain מ \LRE{$x=y$}.
\item אי-שיוויון המשולוש מתקיים: לכל \LRE{$x,y,z\in X$} מתקיים \LRE{$\rho\left(x,z\right)\leq\rho\left(x,y\right)+\rho\left(z,y\right)$}.
\end{enumerate}
\end{definition}
%

\subsection{מטריקת נורמת \LRE{$p$}:}

בהינתן מרחב \LRE{$\mathbb{R}^{d}$}נגדיר את המטריקה \LRE{$\left\Vert \cdot\right\Vert _{p}=\rho:\mathbb{R}^{d}\times\mathbb{R}^{d}\rightarrow\mathbb{R}$}
, \LRE{$p\in\mathbb{N}$}, להיות \LRE{
\begin{gather*}
\left\Vert x,y\right\Vert _{p}=\left(\sum_{k=1}^{d}\left|x_{k}-y_{k}\right|^{p}\right)^{\frac{1}{p}}
\end{gather*}
}

\paragraph{דוגמאות: }

למשל עבור \LRE{$d=3$}, \LRE{$p=2$} נקבל את הנורמה הקלאסית על המרחב
\LRE{
\[
\left\Vert x,y\right\Vert _{2}=\left(\sum_{k=1}^{3}\left|x_{k}-y_{k}\right|^{2}\right)^{\frac{1}{2}}=\sqrt{\left|x_{1}-y_{1}\right|^{2}+\left|x_{2}-y_{2}\right|^{2}+\left|x_{3}-y_{3}\right|^{2}}
\]
}ועבור \LRE{$d=1$}, \LRE{$p=1$} נקבל את הנורמה הקלאסית על ישר\LRE{
\[
\left\Vert x,y\right\Vert _{1}=\left(\sum_{k=1}^{1}\left|x_{k}-y_{k}\right|^{1}\right)^{\frac{1}{1}}=\left|x_{1}-y_{1}\right|
\]
}

\paragraph{דוגמאות למעגל היחידה עבור נורמות שונות ב \LRE{$\mathbb{R}^{2}$}:}
\begin{itemize}
\item \LRE{$\left\Vert x\right\Vert _{1}=\sum_{i=0}^{n}\left|x_{i}\right|$}\\
\begin{tikzpicture}

\draw[->] (-2,0) -- (2,0);
\draw[->] (0,-2) -- (0,2);

\draw[rotate around={45:(0,0)}] (-1,-1) rectangle (1,1);

\end{tikzpicture}
\item \LRE{$\left|x\right|_{2}=\sqrt{\sum_{i=0}^{n}\left|x_{i}\right|^{2}}$}\\
\begin{tikzpicture}

\draw[->] (-2,0) -- (2,0);
\draw[->] (0,-2) -- (0,2);

\draw (0,0) circle (1);

\end{tikzpicture}
\item \LRE{$\left\Vert x\right\Vert _{\infty}=\max_{i\in n}\left|x_{i}\right|$}\\
\begin{tikzpicture}

\draw[->] (-2,0) -- (2,0);
\draw[->] (0,-2) -- (0,2);

\draw (-1,-1) rectangle (1,1);

\end{tikzpicture}
\end{itemize}

\subsection{מטריקות על העתקות}

ניתן להגדיר מטריקות על העתקות בצורה דומה למטריקות על נקודות. למשל
נגדיר את המטריקה הקלאסית על העתקות:
\begin{definition}
\textit{המטריקה הקלאסית על העתקות לינאריות }

נניח מטריקה \LRE{$\rho$} כלשהי על קבוצה \LRE{$X$} ונסמן \LRE{$\left|x\right|=\rho\left(x,\bar{0}\right)$},
\LRE{$x\in X$}. נגדיר את ה\textit{מטריקה הקלאסית על העתקה ליניארית}
\LRE{$a$} המיוצגת ע\textquotedblplain י המטריצה \LRE{$A$} בצורה
הבאה:\LRE{
\[
\left|a\right|=\left\Vert A\right\Vert \coloneqq\max_{x\ne\bar{0}}\frac{\left\Vert Ax\right\Vert }{\left\Vert x\right\Vert }
\]
}
\end{definition}
או במילים, הגודל של ההעתקה \LRE{$a$} הוא השינוי בגדול ביותר המתקבל
ע\textquotedblplain י הפעלת הההעתקה \LRE{$a$} על איבר כלשהו \LRE{$x$}.
כלומר, אם ניקח את כל האיברים בקבוצה \LRE{$X$} ונפעיל על כולם את ההעתקה
\LRE{$a$}, אז לכל איבר היחס בין האיבר המתקבל לבין האיבר המקורי הוא
\LRE{$\frac{\left|Ax\right|}{\left|x\right|}$}, והמקסימום בין כל
היחסים האלו מוגדר להיות הגודל של ההעתקה.

\subsubsection{נורמה לוגריתמית (\textenglish[variant=american]{matrix measure})}

הנורמה הלוגרתמית נותנת לנו מידע על השינוי (שיפוע) של ההעתקה במרחב.
\begin{definition}
\textit{נורמה לוגרתמית }

תהי \LRE{$A$} מטריצה ריבועית ו- \LRE{$\left\Vert \cdot\right\Vert $}
מטריקה על העתקה ליניארית. נגדיר את \textit{הנורמה הלוגרתמית} המתאימה
לה\LRE{
\[
\mu\left(A\right)=\lim_{h\rightarrow0^{+}}\frac{\left\Vert I+hA\right\Vert -1}{h}
\]
}כאשר \LRE{$I$} היא מטריצת היחידה ו- \LRE{$0<h\in\mathbb{R}$}.
\end{definition}
\begin{english}[variant=american]%
\url{https://en.wikipedia.org/wiki/Logarithmic_norm}\texthebrew{למידע
נוסף }
\end{english}%

\section{קונטרקטביליות ויציבות}


\paragraph{מוטיבציה}

קשה מאוד עד בלתי אפשרי לחשב ע\textquotedblplain ע עצמיים של מטריצות
גדולות אבל את \LRE{$\mu$} קל יותר לחשב. אי השוויון נותן לנו דרך לחסום
את הנורמה של ווקטורי מצב לפי הזמן וכך לתת לנו מידע על ההתנהגות של
וקטורים אלו.

\subsection{אי שוויון קופל (\textenglish[variant=american]{coppel's inequality})}

תהי מע' משוואות מצב עם ווקטור \LRE{$\bar{x}$}. נסמן \LRE{
\begin{eqnarray*}
\bar{x}=f\left(x\right), & J\left(\bar{x}\right)=\frac{\partial\bar{f}}{\partial\bar{x}}, & x\left(0\right)=a
\end{eqnarray*}
}כאשר \LRE{$J$} הוא היעקוביאן.

אם ניקח תנאי התחלה \LRE{$a^{1},a^{2}$} בתחום \LRE{$D\subseteq\mathbb{R}^{n}$}
אז\LRE{
\[
\left|\bar{x}\left(t,a^{1}\right)-\bar{x}\left(t,a^{2}\right)\right|\leq e^{\int_{0}^{t}\mu\left(J\left(\alpha\right)\right)d\alpha}\left\Vert a^{1}-a^{2}\right\Vert 
\]
}כאשר עבור הפשטות ניתן לקרב ע\textquotedblplain י \textbf{חסם עליון}\LRE{
\[
e^{\int_{0}^{t}\mu\left(J\left(\alpha\right)\right)d\alpha}\approx e^{t\cdot\max_{x\in\mathbb{R}^{n}}\mu\left(J\left(\bar{x}\right)\right)}
\]
}
\begin{definition}
\textit{קונטרקטביליות}

מערכת משוואות מצב תיקרא \textit{קונטרקטבילית} בתחום \LRE{$D\subseteq\mathbb{R}^{n}$}
אם מתקיימים התנאים הבאים:
\begin{itemize}
\item מתקיים אי-שיוויון קופל.
\item הנורמה הלוגרתמית שלילית (\LRE{$\mu<0$}) בתחום \LRE{$D$}.
\end{itemize}
\end{definition}
לכן כדי למצוא קורדינאטות נדרוש \LRE{$\mu<0$} בכל התחום \LRE{$D$}.


\subsection{קונטרקטיביליות ויציבות}
\begin{definition}
\textit{יציבות של מערכת קונטרקטבילית}

מערכת משוואות מצב קונטרקטבילית בתחום \LRE{$D$} \textit{יציבה} אם
יש בתחום \LRE{$D$} בדיוק נקודת שיווי משקל יחידה.
\end{definition}

\subsubsection{מציאת תחום היציבות עבור מערכת משוואות מצב}

בהינתן מערכת משוואות מצב במרחב \LRE{$\mathbb{R}^{n}$} נרצה למצוא
תחום בו המערכת היא קונטרקטבילית ויציבה. דרגת החופש שלנו היא בחירת
המטריקה המגדירה את התחום. ניתוח זה מאפשר לנו למצוא תנאים על משתני
וקטור המצב עבורם המערכת יציבה.

נתמקד בנורמות \LRE{$L_{1},L_{2},L_{\infty}$} שהן השימושיות ופשוטות
לחישוב בדרך כלל:
\begin{enumerate}
\item מטריקת \LRE{$L_{1}$}:\LRE{
\begin{eqnarray*}
\left\Vert x\right\Vert _{1}=\sum_{i=0}^{n}\left|x_{i}\right| & \Rightarrow & \mu_{1}\left(A\right)=\max_{j}\left\{ A_{jj}+\sum_{i=1,i\ne j}^{n}\left|A_{ij}\right|\right\} 
\end{eqnarray*}
}כאשר \LRE{$i$} אינדקס השורות ו-\LRE{$j$} אינדקס העמודות של המטריצה
\LRE{$A$}. 
\item מטריקת \LRE{$L_{2}$}:\LRE{
\begin{eqnarray*}
\left\Vert x\right\Vert _{2}=\sqrt{\sum_{i=0}^{n}\left|x_{i}\right|^{2}} & \Rightarrow & \mu_{2}\left(A\right)=\lambda_{max}\left(\frac{A+A^{T}}{2}\right)
\end{eqnarray*}
}
\item מטריקת \LRE{$L_{\infty}$}:\LRE{
\begin{eqnarray*}
\left\Vert x\right\Vert _{\infty}=\max_{i\in n}\left|x_{i}\right| & \Rightarrow & \mu_{\infty}\left(A\right)=\max_{i}\left\{ A_{ii}+\sum_{j=1,j\ne i}^{n}\left|A_{ij}\right|\right\} 
\end{eqnarray*}
}אשר \LRE{$i$} אינדקס השורות ו-\LRE{$j$} אינדקס העמודות של המטריצה
\LRE{$A$}. 
\end{enumerate}
הערה: כאשר \LRE{$\mu=0$} אמנם אין התכנסות לנקודת שיווי המשקל אך המרחק
בין וקטור המצב לנקודת שיווי המשקל קבוע. מצב זה אינו רצוי בדרך כלל
אך גם אינו מתבדר.

\paragraph{דוגמא}

נניח מערכת הנתונה ע\textquotedblplain י משוואות המצב\LRE{
\begin{align*}
 & \frac{dx_{1}}{dt}=-x_{1}-x_{1}x_{2}\\
 & \frac{dx_{2}}{dt}=x_{1}+x_{2}
\end{align*}
}נק' ש\textquotedblplain מ של המערכת הן \LRE{$\left(0,0\right),\left(-1,-1\right)$}.
יש יותר מנקודת שיווי משקל אחת ולכן המערכת אינה קונטרקטבילית בכל מרחב
המצב.

נחשב את היעקוביאן של המערכת\LRE{
\begin{gather*}
J\left(\bar{x}\right)=\frac{\partial\bar{f}}{\partial\bar{x}}=\left[\begin{array}{cc}
-1-x_{2} & -x_{1}\\
1 & -1
\end{array}\right]
\end{gather*}
}כעת נחשב את \LRE{$\mu\left(J\left(\bar{x}\right)\right)$} עבור המטריקות
\LRE{$L_{1},L_{2},L_{\infty}$} ונבדוק אם קיים תחום בו \LRE{$\mu\left(J\left(\bar{x}\right)\right)<0$}:
\begin{enumerate}
\item \LRE{$L_{1}$}: \LRE{
\[
\mu_{1}\left(J\right)=\max_{j}\left\{ A_{jj}+\sum_{i=1,i\ne j}^{n}\left|A_{ij}\right|\right\} =\max\left\{ -1-x_{2}+1,-1+\left|x_{1}\right|\right\} =\max\left\{ -x_{2},-1+\left|x_{1}\right|\right\} 
\]
}נקבל שני תחומים בהם \LRE{$\mu_{1}<0$} כלומר תחומים בהם המערכת קונטרקטבילית:
\begin{itemize}
\item \LRE{$0>\mu_{1}=-x_{2}>-1+\left|x_{1}\right|$} 
\item \LRE{$-x_{2}<-1+\left|x_{1}\right|=\mu_{1}<0$}
\end{itemize}
בשני המקרים נקודות הש\textquotedblplain מ נמצאות מחוץ לתחום ולכן לא
מובטח לנו יציבות.
\item \LRE{$L_{2}$}:\LRE{
\[
\mu_{2}\left(J\right)=\lambda_{max}\left(\frac{J+J^{T}}{2}\right)=\lambda_{max}\left[\begin{array}{cc}
-1-x_{2} & \frac{1-x_{1}}{2}\\
\frac{1-x_{1}}{2} & -1
\end{array}\right]
\]
}הערכים העצמיים של המטריצה מקיימים\LRE{
\[
\left|\begin{array}{cc}
\lambda+x_{2}+1 & \frac{x_{1}-1}{2}\\
\frac{x_{1}-1}{2} & \lambda+1
\end{array}\right|=\left(\lambda+x_{2}+1\right)\left(\lambda+1\right)-\left(\frac{x_{1}-1}{2}\right)^{2}=\lambda^{2}+\left(x_{2}+2\right)\lambda+x_{2}+1-\left(\frac{x_{1}-1}{2}\right)^{2}=0
\]
}אנחנו מחפשים \LRE{$\lambda$} מקסימלי כך ש \LRE{$\mu_{2}\left(J\right)<0$}
כלומר \LRE{$\lambda_{max}<0$}. נשתמש במבחן \LRE{$RH$} כדי למצוא
תנאי על \LRE{$x_{1},x_{2}$} עבורו זה ייתקיים\LRE{
\begin{align*}
\lambda_{2} &  &  & 1 &  & x_{2}+1-\left(\frac{x_{1}-1}{2}\right)^{2}\\
\lambda_{1} &  &  & x_{2}+2 &  & 0\\
\lambda_{0} &  &  & x_{2}+1-\left(\frac{x_{1}-1}{2}\right)^{2} &  & 0
\end{align*}
}כלומר נדרוש\LRE{
\begin{eqnarray*}
x_{2}+2>0 & \Rightarrow & x_{2}>-2
\end{eqnarray*}
}וגם\LRE{
\begin{gather*}
x_{2}+1-\left(\frac{x_{1}-1}{2}\right)^{2}>0\\
4x_{2}+4-x_{1}^{2}+2x_{1}-1>0\\
\Downarrow\\
4x_{2}>x_{1}^{2}-2x_{1}-3
\end{gather*}
}כלומר התחום מעל פרבולה החותכת את ציר \LRE{$x_{1}$} בנקודות \LRE{$-1,3$}
ונקודת המינימום שלה היא \LRE{$\left(1,-1\right)$}, בתחום זה המערכת
היא קונטרקטבילית. אנחנו מחפשים תחום שבו המערכת תהיה יציבה לכן נדרוש
בנוסף שהמרחק של וקטור בתחום מנקודת שיווי המשקל יהיה קטן מהמרחק הקצר
ביותר מנקודת השיווי משקל \LRE{$\left(0,0\right)$} לפרבולה.\\
נקודה כללית על הפרבולה היא \LRE{$\bar{x}_{p}=\left(x_{1},\frac{x_{1}^{2}-2x_{1}-3}{4}\right)$}
לכן מרחק נקודה כללית על הפרבולה מהראשית היא\LRE{
\[
\left|\bar{x}_{p}-\bar{0}\right|=\sqrt{\left(x_{1}\right)^{2}+\left(\frac{x_{1}^{2}-2x_{1}-3}{4}\right)^{2}}
\]
} המרחק המינימלי הוא \LRE{$d_{min}\approx0.647$} ומתקבל עבור \LRE{$x_{1}\approx-0.364$}.\\
מרחק זה מגדיר לנו מעגל \LRE{$D$} (כי הנורמה היא \LRE{$L_{2}$}) ברדיוס
\LRE{$d_{min}$} סביב נקודת השיווי משקל \LRE{$a_{1}=\bar{0}$} כל
שלכל וקטור התחלה \LRE{$a_{2}$} מתקיים\LRE{
\[
\left|\bar{x}\left(t,a^{1}\right)-\bar{x}\left(t,a^{2}\right)\right|\leq e^{\int_{0}^{t}\mu_{2}\left(J\left(\alpha\right)\right)d\alpha}\left|a^{1}-a^{2}\right|\leq e^{\int_{0}^{t}\mu_{2}\left(J\left(\alpha\right)\right)d\alpha}d_{min}
\]
}כאשר \LRE{$\mu_{2}<0$} כלומר כל וקטור מצב בתחום \LRE{$D$} יתכנס
בהכרח לנקודת השיווי משקל ולכן בתחום \LRE{$D$} המערכת יציבה.
\item \LRE{$L_{\infty}$}:\LRE{
\[
\mu_{\infty}\left(J\right)=\max_{i}\left\{ A_{ii}+\sum_{j=1,j\ne i}^{n}\left|A_{ij}\right|\right\} =\max\left\{ -1-x_{2}+\left|x_{1}\right|,0\right\} \geq0
\]
}תמיד אי-שלילי ולכן אין תחום קונטרקטבילי.
\end{enumerate}

\section{תוספת}

\subsection{שימוש בדימיון מטריצות על מנת להגידר נורמה אחרת}

\LRE{$P$} מטריצה הפיכה. נגדיר נורמה\LRE{
\[
\left|x\right|_{P}:=\left|Px\right|
\]
}ו-\LRE{
\[
\mu_{P}\left(A\right)=\mu\left(P^{-1}AP\right)
\]
}

.

\subsection{קריטריון ליאפונוב}

מערכת משוואות ממצב ליניאריות\LRE{
\[
\dot{x}=Ax
\]
}מערכת עם \LRE{$A$}יציבה אם\LRE{
\[
\forall Q,P>0:A^{T}P+PA=-Q
\]
}

ניתן גם להניח \LRE{$Q$}, למשל \LRE{$Q=I$}, ואז לפתור מערכת משוואות
ולמצוא \LRE{$P$} (\LRE{$P$} תהיה סימטרית) .


\subsection{תיקון הגדרה לקונטרקטיביות:}

מערכת דיגימת\LRE{
\[
\dot{x}=f\left(x\right)
\]
}\LRE{
\[
a,b\in R^{b}
\]
} מע נקראת קונטרקטבילית אם \LRE{
\[
\left|x\left(t,a\right)-x\left(t,b\right)\right|\leq e^{-\eta t}\left|a-b\right|
\]
}


\paragraph{משפט\textmd{ }3}

אם נתונה מע' \LRE{$f$} ומתקיים\LRE{
\[
\mu\left(J\left(x\right)\right)\leq\eta<0,\forall x\in R^{n}
\]
}אפילו כאשר נתון תחום \LRE{$\Omega$} של \LRE{$R^{n}$} עם תנאי התחלה
על התחום

\newpage{}

\section{משוואות מצב של מערכות הספק}

הערה: בקורס למדנו את מערכת המשוואות לאחר קירוב \textenglish[variant=american]{dc
power flow}, פה נרצה את המשוואות ללא קירוב.

\subsection{משוואות זרימת ההספק{\normalsize{} }\textenglish[variant=american]{power
flow equations} ומשוואת{\normalsize{} }\textenglish[variant=american]{swing}}

בהינתן מערכת הספק בעלת \LRE{$N$} פסי צבירה, ההספק של פסי הצבירה \LRE{$i$}
נתון ע\textquotedblplain י \LRE{
\[
P_{i}=\sum_{n=1}^{N}\left|V_{i}\right|\left|V_{n}\right|\left|y_{i,n}\right|\cos\left(\delta_{n}-\delta_{i}+\theta_{i,n}\right)+P_{L,i}
\]
}\LRE{$\forall i=1,\dots,n$}\LRE{
\[
Q_{i}=\sum_{n=1}^{N}\left|V_{i}\right|\left|V_{n}\right|\left|y_{i,n}\right|\sin\left(\delta_{n}-\delta_{i}+\theta_{i,n}\right)+Q_{L,i}
\]
}

כאשר:
\begin{itemize}
\item \LRE{$P_{i}$},\LRE{$Q_{i}$} - הספק אקטיבי וריאקטיבי של הגנרטור 
\item \LRE{$\left|V_{i}\right|$}- אמפליטודת המתח 
\item \LRE{$\delta_{i}$} - פאזת המתח
\item \LRE{$\left|y_{i,n}\right|$}, \LRE{$\theta_{i,n}$} - גודל וזווית
אלמנט \LRE{$i,n$} מטריצת האדמינטנסים
\item \LRE{$P_{L,i}$},\LRE{$Q_{L,i}$} - הספק אקטיבי וריאקטיבי של עומס 
\end{itemize}
בנוסף לכל גנרטור \LRE{$i$}, הדינמיקה של הגנרטור מתוארת ע\textquotedblplain י
\LRE{
\[
\frac{d}{dt}\omega_{i}=K\left(3P_{ref}-3P_{i}-\frac{1}{D}\left(\omega_{i}-\omega_{s}\right)\right)
\]
}

כאשר:
\begin{itemize}
\item \LRE{$P_{ref}$} - ההספק הנדרש מהגנרטור (פאזה אחת)
\item \LRE{$P_{i}$} - ההספק בפועל מהגנרטור לפי משוואות זרימת ההספק
\item \LRE{$D$} - מקדם ריסון
\item \LRE{$\omega$} - תדר הגנרטור
\item \LRE{$\omega_{s}$} - תדר העבודה של המערכת
\end{itemize}
בנוסף מתקיים הקשר בין תדר הגנרטור לפאזת המתח שלו\LRE{
\[
\frac{d}{dt}\delta_{i}=\omega_{i}-\omega_{r}
\]
}כאשר:
\begin{itemize}
\item \LRE{$\omega_{r}$} - תדר הייחוס של המערכת. נהוג לבחור את תדר הייחוס
כך ש- \LRE{$\delta_{1}=0$}.
\end{itemize}
יחדיו, משוואות אלו מאפיינות לנו מערכת מצב של המערכת שהמשתני המצב שלה
הם (ומתארים את האנרגיה במערכת) \LRE{
\[
\bar{x}=\left\{ \delta_{i},\omega_{i}:\forall i=1,\dots,n\right\} 
\]
}

\subsection{מודל חסר הפסדים}

נניח את ההנחות הבאות כדי לקבל מודל פשוט (בדומה לסיכום על \LRE{$N$}
גנרטורים).
\begin{itemize}
\item רשת ההולך אינה מכילה עומסים. כלומר כל העומסים מחוברים לגנרטור כלשהו
במערכת.
\item רשת ההולכה חסרת הפסדים ובעלת השראות בלבד.
\item אין עומסים ריאקטיביים.
\end{itemize}
עבור הנחות אלו נקבל\LRE{
\begin{eqnarray*}
\left|y_{i,n}\right|=\left|y_{n,i}\right|, & \forall i:\,\theta_{i,i}=-\frac{\pi}{2}, & \forall i\neq j:\,\theta_{i,j}=\frac{\pi}{2}
\end{eqnarray*}
}וכתוצאה מכך \LRE{$\forall i=1,\dots,n$}\LRE{
\begin{align*}
 & P_{i}=\sum_{n=1}^{N}a_{i,n}\sin\left(\delta_{i}-\delta_{n}\right)+P_{L,i}\\
 & Q_{i}=0
\end{align*}
}כאשר:
\begin{itemize}
\item \LRE{$a_{i,n}=\left|V_{i}\right|\left|V_{n}\right|\left|y_{i,n}\right|>0$},
\LRE{$a_{i,n}=a_{n,i}$}
\item בנקודות שיווי המשקל של המערכת כל תדרי המערכת שווים \LRE{$\forall i:\omega_{i}=\omega_{1}=\omega_{s}$}
\end{itemize}

\subsubsection{גנרטור ורשת קשיחה}

נניח גנרטור יחיד עם חיבור לרשת קשיחה בתדר \LRE{$\omega_{1}=\omega_{s}$}
עם מתח \LRE{$V\left(t\right)=\left|V_{1}\right|\angle0^{\circ}$}
(\LRE{$\delta_{1}=0$}). הרשת הקשיחה הינה מקור אידיאלי ואינה משתנה
ולכן מכתיבה ייחוס עבור התדר \LRE{$\omega_{r}=\omega_{s}$}. כיוון
שהנחנו שאין הפסדים ברשת ההולכה, הרשת הקשיחה קובעת את המתח לכל פסי
הצבירה \LRE{$\forall i:\left|V_{i}\right|=\left|V_{1}\right|$}.

נסמן את הגנרטור ב- \LRE{$i=2$}. נקבל את המשוואות\LRE{
\begin{align*}
 & \delta_{1}=0\\
 & \frac{d}{dt}\omega_{2}=K\left(3P_{ref}-3P_{2}-\frac{1}{D}\left(\omega_{2}-\omega_{s}\right)\right)\\
 & \frac{d}{dt}\delta_{2}=\omega_{2}-\omega_{s}\\
 & P_{2}=\sum_{n=1}^{N}a_{2,n}\sin\left(\delta_{2}-\delta_{n}\right)+P_{L,2}=a_{2,1}\sin\left(\delta_{2}-\delta_{1}\right)+P_{L,2}
\end{align*}
} משוואות המצב של המערכת הן\LRE{
\[
\begin{cases}
\frac{d}{dt}\delta_{2}=\omega_{2}-\omega_{s}\\
\frac{d}{dt}\omega_{2}=3K\left(P_{ref}-P_{L,2}\right)-3Ka_{2,1}\sin\left(\delta_{2}\right)-\frac{K}{D}\left(\omega_{2}-\omega_{s}\right)
\end{cases}
\]
}נחשב את היעקוביאן של המערכת\LRE{
\[
J=\left[\begin{array}{cc}
0 & 1\\
-3Ka_{2,1}\cos\left(\delta\right) & -\frac{K}{D}
\end{array}\right]
\]
}

\subsubsection{2 גנרטורים}

נבחר ייחוס \LRE{$\omega_{1}=\omega_{r}$} - כרגע לא מתעסקים
\end{document}
